\section{Description}
Nous représentons les cartes sous forme de graphes orientés.
Les Noeuds correspondent à des intersections ou à des culs de sac.
Les Noeuds ne possèdent aucune pondération.
Les Noeuds sont connectés entre eux via des arcs pondérés. les pondérations sont de différents
types:
\begin{itemize}
  \item le kilométrage
  \item la dangerosité
  \item la pente
  \item l'attrait touristique
\end{itemize}
L'Utilisateur aura le choix entre plusieurs mode de fonctionnements:
\begin{itemize}
  \item Tracer un trajet a la main (i.e. il indique les points par lesquels il veux passer, et l'algorithme trace les plus courts chemin entre ces points)
  \item Faire trouver à l'application une boucle (i.e. l'utilisateur veux se promener, il configure alors ses préférences et l'algorithme cherche à coller au mieux à ces préférences)
  \item Faire trouver à l'application un trajet d'un point A à B (i.e. Fonction de GPS classique, mais l'utilisateur pourra aisément choisir ce qu'il préfère, exemple, je préfère faire un chemin un peu plus long mais uniquement sur piste cyclable).
\end{itemize}
\section{Modélisation}
